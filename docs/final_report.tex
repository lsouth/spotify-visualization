\documentclass[journal]{vgtc}                % final (journal style)
% \documentclass[review,journal]{vgtc}         % review (journal style)
%\documentclass[widereview]{vgtc}             % wide-spaced review
% \documentclass[preprint,journal]{vgtc}       % preprint (journal style)

% This is eysa's macro file
\usepackage[usenames,dvipsnames]{xcolor}

\newcommand{\eysa}[1]{\textcolor{OliveGreen}{eysa: #1}}

%% Uncomment one of the lines above depending on where your paper is
%% in the conference process. ``review'' and ``widereview'' are for review
%% submission, ``preprint'' is for pre-publication, and the final version
%% doesn't use a specific qualifier.

%% Please use one of the ``review'' options in combination with the
%% assigned online id (see below) ONLY if your paper uses a double blind
%% review process. Some conferences, like IEEE Vis and InfoVis, have NOT
%% in the past.

%% Please note that the use of figures other than the optional teaser is not permitted on the first page
%% of the journal version.  Figures should begin on the second page and be
%% in CMYK or Grey scale format, otherwise, colour shifting may occur
%% during the printing process.  Papers submitted with figures other than the optional teaser on the
%% first page will be refused. Also, the teaser figure should only have the
%% width of the abstract as the template enforces it.

%% These few lines make a distinction between latex and pdflatex calls and they
%% bring in essential packages for graphics and font handling.
%% Note that due to the \DeclareGraphicsExtensions{} call it is no longer necessary
%% to provide the the path and extension of a graphics file:
%% \includegraphics{diamondrule} is completely sufficient.
%%
\ifpdf%                                % if we use pdflatex
  \pdfoutput=1\relax                   % create PDFs from pdfLaTeX
  \pdfcompresslevel=9                  % PDF Compression
  \pdfoptionpdfminorversion=7          % create PDF 1.7
  \ExecuteOptions{pdftex}
  \usepackage{graphicx}                % allow us to embed graphics files
  \DeclareGraphicsExtensions{.pdf,.png,.jpg,.jpeg} % for pdflatex we expect .pdf, .png, or .jpg files
\else%                                 % else we use pure latex
  \ExecuteOptions{dvips}
  \usepackage{graphicx}                % allow us to embed graphics files
  \DeclareGraphicsExtensions{.eps}     % for pure latex we expect eps files
\fi%

%% it is recomended to use ``\autoref{sec:bla}'' instead of ``Fig.~\ref{sec:bla}''
\graphicspath{{figures/}{pictures/}{images/}{./}} % where to search for the images

\usepackage{microtype}                 % use micro-typography (slightly more compact, better to read)
\PassOptionsToPackage{warn}{textcomp}  % to address font issues with \textrightarrow
\usepackage{textcomp}                  % use better special symbols
\usepackage{mathptmx}                  % use matching math font
\usepackage{times}                     % we use Times as the main font
\renewcommand*\ttdefault{txtt}         % a nicer typewriter font
\usepackage{cite}                      % needed to automatically sort the references
\usepackage{tabu}                      % only used for the table example
\usepackage{booktabs}                  % only used for the table example
%% We encourage the use of mathptmx for consistent usage of times font
%% throughout the proceedings. However, if you encounter conflicts
%% with other math-related packages, you may want to disable it.

%% In preprint mode you may define your own headline.
%\preprinttext{To appear in IEEE Transactions on Visualization and Computer Graphics.}

%% If you are submitting a paper to a conference for review with a double
%% blind reviewing process, please replace the value ``0'' below with your
%% OnlineID. Otherwise, you may safely leave it at ``0''.
\onlineid{0}

%% Paper title.
\title{My Week in Music: Visualizing Music Listening History}
%% This is how authors are specified in the journal style

%% indicate IEEE Member or Student Member in form indicated below
\author{Sara Di Bartolomeo, Eysa Lee, Amogh Pradeep, Laura South}
\authorfooter{
%% insert punctuation at end of each item
\item
 Sara Di Bartolomeo, Eysa Lee, Amogh Pradeep, and Laura South are with Northeastern University. E-mail: \{dibartolomeo.s\,$|$\,lee.ey\,$|$\,pradeep.am\,$|$\,south.l\}@husky.neu.edu.
}

%other entries to be set up for journal
\shortauthortitle{Di Bartolomeo \MakeLowercase{\textit{et al.}}: My Week in Music: Visualizing Music Listening History}
%\shortauthortitle{Firstauthor \MakeLowercase{\textit{et al.}}: Paper Title}

%% Abstract section.
\abstract{\eysa{todo}%
} % end of abstract

%% Keywords that describe your work. Will show as 'Index Terms' in journal
%% please capitalize first letter and insert punctuation after last keyword
\keywords{\eysa{todo}}

%% ACM Computing Classification System (CCS). 
%% See <http://www.acm.org/class/1998/> for details.
%% The ``\CCScat'' command takes four arguments.

% \CCScatlist{ % not used in journal version
%  \CCScat{K.6.1}{Management of Computing and Information Systems}%
% {Project and People Management}{Life Cycle};
%  \CCScat{K.7.m}{The Computing Profession}{Miscellaneous}{Ethics}
% }

%% Uncomment below to include a teaser figure.
\teaser{
  \centering
  \includegraphics[width=\linewidth]{mockup}
  \caption{A week in music: \eysa{Tentative screenshot of our thing}.}
	\label{fig:teaser}
}

%% Uncomment below to disable the manuscript note
%\renewcommand{\manuscriptnotetxt}{}

%% Copyright space is enabled by default as required by guidelines.
%% It is disabled by the 'review' option or via the following command:
\nocopyrightspace

\vgtcinsertpkg

%%%%%%%%%%%%%%%%%%%%%%%%%%%%%%%%%%%%%%%%%%%%%%%%%%%%%%%%%%%%%%%%
%%%%%%%%%%%%%%%%%%%%%% START OF THE PAPER %%%%%%%%%%%%%%%%%%%%%%
%%%%%%%%%%%%%%%%%%%%%%%%%%%%%%%%%%%%%%%%%%%%%%%%%%%%%%%%%%%%%%%%%

\begin{document}

%% The ``\maketitle'' command must be the first command after the
%% ``\begin{document}'' command. It prepares and prints the title block.

%% the only exception to this rule is the \firstsection command
\firstsection{Introduction}

\maketitle

%% \section{Introduction} %for journal use above \firstsection{..} instead
\eysa{This is the template for the paper and required sections (as specified in the milestone document). Text and other sections will be added.\\
Our motivation is roughly that we were curious about our own listening history and knew the data was there}.

\section{Related Works}
\eysa{Related works include Spotify's yearly end of year review and the LastFM document we found early on}
\subsection{Spotify Year in Review}

Introduce SpotifyWrapped (and/or year in review). It gives aggregate data (total minutes listened, top songs, top artists, top genre).\cite{Spo18} \eysa{A small note is that the card this is referenced is no longer available.}

\subsection{Last.fm}

\eysa{Introduce Last.fm, which lets users explore their lisetning data by either connecting with their streaming service or uploading. There has much work in using this data}

Pretzlav created Last.fm Explorer \cite{Pre08} \eysa{either add screenshots or describe in words. It also has a stacked bar chart}.

Later, Dang, Anand, and Wilkinson created FmFinder \cite{Dan12} \eysa{Describe this one, too}

More recently Boeing explore their own listening history from Last.fm and wrote up a blog post \cite{Boe16}. \eysa{Make this sound nicer}

\section{Process}
\subsection{Task Analysis}
We conducted nine short interviews with potential users of our visualization \eysa{todo: add summary of the results of the interviews. Can add a table if I reaaaaally want to...}

From our interviews, we determined our domain goals to be the following:
\begin{enumerate}
  \item View daily listening history split by genre
  \item View recent top artists/tracks
  \item Compare listening history between users
\end{enumerate}

\eysa{todo: add discussion about these domain goals}

\begin{figure*}
\centering
\mbox{
  \subfigure[Genre interaction.\label{fig:genre_interaction}]{\includegraphics[width=0.3\textwidth]{genre_interaction}}\quad
  \subfigure[Artist interaction.\label{fig:artist_interaction}]{\includegraphics[width=0.3\textwidth]{artist_interaction}}\quad
  \subfigure[Track interaction.\label{fig:track_interaction}]{\includegraphics[width=0.3\textwidth]{track_interaction}}
}
\caption{Primary interactions: Hovering over different areas in the visualization highlights and connects data between the stacked area chart and the two lists.\eysa{come back and make this nicer to read}}
\label{fig:interactions}
\end{figure*}

\section{Design}

Interactions can be found in \autoref{fig:interactions}.

This section will address:
\begin{enumerate}
  \item How do we address our domain goals from before.
  \item Usability testing. Was there anything we changed based on classmates testing our preliminary design
\end{enumerate}

\section{Discussion}
This section will include discussion on:
\begin{enumerate}
  \item What was the response like?
  \item What were areas we wanted to explore but couldn't
  \item What could we improve?
\end{enumerate}

\section{Conclusion}

Future directions include:
\begin{enumerate}
  \item Computing over the data
  \item How can we make it easier for people to compare?
\end{enumerate}

%% if specified like this the section will be committed in review mode
\acknowledgments{
The authors wish to thank the Spring 2019 teaching staff of Michelle Borkin's CS 7250 class for their helpful comments and guidance.
}

%\bibliographystyle{abbrv}
\bibliographystyle{abbrv-doi}
%\bibliographystyle{abbrv-doi-narrow}
%\bibliographystyle{abbrv-doi-hyperref}
%\bibliographystyle{abbrv-doi-hyperref-narrow}

\bibliography{references}
\end{document}

